\section{Funding}

\subsection{Research Awards/Grants}

\cvline{NSF OAC-2107020}{PI, Collaborative Research: OAC Core: Advancing Low-Power Computer Vision at the Edge, \$250,000, 2021-2024}

\cvline{NSF OAC-2104319}{PI, CDSE: Collaborative: Cyber Infrastructure to Enable Computer Vision Applications at the Edge Using Automated Contextual Analysis, \$174,749, 2021-2024}

\cvline{NSF HRD-2121654}{Senior Personnel, ADVANCE Adaptation: INSPIRED-Inclusive Practices in the Retention and Equity of Diverse Faculty, \$972,496, 2021-2024}

\cvline{NSF OAC-2027514}{Senior Personnel/Consultant, Collaborative: RAPID: Leveraging New Data Sources to Analyze the Risk of COVID-19 in Crowded Locations, \$50,000, 2020-2021}

\cvline{NSF CNS-1738691}{Co-PI, Collaborative Research: Chicago Alliance for Equity in Computer Science, \$72,497, 2017-2021}

\cvline{NSF OAC-1445347}{PI, EAGER: Collaborative Research: Making Software Engineering Work for Computational Science and Engineering: An Integrated Approach, \$109,598, 2014-2018}

\cvline{NSF CCF-0444197}{PI, Collaborative Proposal: Ultra-Scalable System Software and Tools for Data-Intensive Computing, \$72,433, 2004-2007}

\cvline{NSF CNS-0205652}{PI, ITR: The Community Information Technology Entrepreneurship Project, \$1,034,500, 2002-2006}

\cvline{NSF CNS-10423372}{Co-PI, Collaborative Research: BPC-LSA: ACM SIGBP: Forming an ACM Special Interest Group to Scale the Impact of BPC Activities, \$38,782, 2010-2014}

\cvline{NEH HD-50782-09}{Co-Director, Humanities Research Infrastructure and Tools (HRIT): An Environment for Collaborative Textual Scholarship, \$50,000, 2009-2010}

\cvline{NSF CNS-0837769}{Co-PI, Improving Metropolitan Participation to Accelerate Computing Throughput and Success (IMPACTS), \$141,711, 2008-2009}

\cvline{NSF DBI-0552888}{Co-PI, REU Site: Integrated Cross-Disciplinary Summer Program in Bioinformatics, \$292,412, 2006-2010}

\cvline{NSF OAC-0228926}{Co-PI, HPNC: HPNC for Science Research at Loyola University Chicago, \$150,000, 2002-2005}

\cvline{Google}{Co-PI, Google Research Awards, Machine Perception, \$75,000, 2017}

\cvline{DARPA SBIR Phase I}{PI, Ad Hoc Human Interaction Networks for Asymmetric Threat Surveillance, \$100,000, 2002}

\cvline{DARPA SBIR Phase I}{Co-PI, Cluster Based Repositories and Analysis, \$100,000, 2003}

\subsection{University Funding}

\cvline{Office of Research Services}{PI, Autonomous Vehicle Security, \$5,000, 2019-2020}

\cvline{Office of Research Services}{PI, Research Stimulation: Chicago Clean Air/Clean Water, \$20,000, 2009-2010}

\cvline{Office of Research Services}{Co-PI, Research Stimulation: Modernist Circles/Textual Studies, \$20,000, 2009-2010}

\subsection{Gifts}

\cvline{Google}{PI, Unrestricted gift to support machine learning and software engineering research at Loyola, \$30,000, 2019}

\cvline{Facebook Research}{PI, Unrestricted gift to support low-power computer vision and reproducibility research at Loyola, \$30,000, 2020}

\cvline{Google Research Awards}{PI, \$75,000, 2017}

\cvline{4C Insights}{PI, Unrestricted gift to support systems research at Loyola, \$10,000, 2017}

\cvline{Typesafe}{Co-PI, Unrestricted gift to support languages/systems research at Loyola, \$5,000, 2015}

\cvline{Microsoft}{PI, Unrestricted gift to support HPC and Bioinformatics Research, \$15,000, 2009}

\cvline{Microsoft}{PI, In-kind equipment donation of computational cluster (80 nodes), \$100,000, 2009}

\cvline{HP}{Co-PI, Learning the Wonders of Computing with Wireless Collaboration, \$68,000, 2007}

\cvline{Hostway Corporation}{PI, Unrestricted gift to support creation of open source laboratory, \$30,000, 2005}




1. K. Läufer, C. Sekharan, and G. K. Thiruvathukal. NVIDIA equipment award for Early Adopters under the NSF/IEEE-TCPP Curriculum Initiative on Parallel and Distributed Computing, January 2012.

2. R.\ Greenberg (PI), W. Honig, K. Läufer, C. Putonti, and G. K. Thiruvathukal. Improving Metropolitan Participation to Accelerate Computing Throughput and Success (IMPACTS). National Science Foundation, Broadening Participation in Computing (BCP) program, Award #0837769, $141,711; 2008-2011.

3. W.\ Honig, K.\ Läufer, and G.\ K.\ Thiruvathukal. Learning the Wonders of Computing with Wireless Collaboration. HP Technology for Teaching Grant #1900784, $68,000; 2007–2008.

4. G.\ K.\ Thiruvathukal (PI), C.\ Sekharan (co-PI), and K.\ Läufer (paid senior personnel). ITR: The Community Information Technology Entrepreneurship Project. National Science Foundation, Information Technology Research (ITR) program, Award #0205652, $1,000,000; 2002–2005.

5. K.\ Läufer (PI) and George K.\ Thiruvathukal. A multi-platform application suite for enhancing South Asian language pedagogy. South Asia Language Resource Center (SALRC) Mini-Grant, $5,000; 2004–2005.

6. R.\ Jagadeesan (PI) and K.\ Läufer. The Triveni Project, National Science Foundation, Software Engineering and Languages program, Award #9901071, $153,988; 1999–2002.

7. P.\ Dordal, R.\ Jagadeesan (PI), K.\ Läufer, and C.\ Sekharan. Sun Microsystems, Academic Equipment Grant, $89,000; 1999.

8. P.\ Dordal, R.\ Jagadeesan, K.\ Läufer (PI), and C.\ Sekharan. Microsoft Instructional Lab Grant, Grant #190, $70,000; 1996–1998.

9. K.\ Läufer. European Union Human Capital and Mobility Grant, August 1995.

Funded Internal Grants

1. G. K. Thiruvathukal (PI), N. Tuchman, J. Frendreis, P. Geddes, and D. Slavsky, Research Stimulation: Chicago Clean Air, Clean Water, $10,000. Renewed in Fall 2009 for an additional $10,000. Key contributor on proposal and ensuing project.

2. K.\ Läufer (PI). Fall 2009 Faculty Fellow of Center for Urban Environmental Research and Policy (CUERP). Award includes $1,500 stipend.

3. K.\ Läufer. Mulcosoft: Programming Languages and Frameworks for Multi-Core Systems, Loyola Research Support Grant, $3,000; 2007.

4. K.\ Läufer. TriveniLite: A Framework for Defining and Composing Tasks in Concurrent Applications, Loyola Summer Research Stipend, $4,000; 2004.

5. K.\ Läufer (PI) and George K.\ Thiruvathukal. Handheld and Wireless Technology in the Classroom: A Concept/Research Laboratory for Teaching South Asian Languages, Loyola College of Arts and Sciences Research Seed Grant, $2,500; 2003.

6. K.\ Läufer. A Compositional Language for Concurrent Programming, Loyola Summer Research Stipend, $6,000; 1998.

7. K.\ Läufer. Loyola Faculty Development Initiative Award. 1998.

8. K.\ Läufer. Type Inference for Objects, Classes, and Patterns, Loyola Summer Research Stipend, $6,000; 1996.

9. K.\ Läufer. Loyola Research Support Grant, $2,000; 1995.

10. K.\ Läufer. Polymorphic Type Inference and Object-Oriented Programming, Loyola Summer Research Stipend, $5,000; 1993.

11. K.\ Läufer. NYU Graduate School of Arts and Sciences Travel Award; 1991.